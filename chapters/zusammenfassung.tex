\chapter{Zusammenfassung}

Während der Zeit im Praxissemseter boten sich viele Möglichkeiten in den realen Arbeitsalltag einzutauchen. Durch das Einarbeiten in eine unbekannte Umgebung und noch nie verwedeten Technologien, gabe es vieles Neues zu lernen. Durch eine kurze Zeit als Werkstudent vor dem Praxissemeter, gab es zwar schon ein gewisses grundverständnis, wie zum Beispiel die Heransgehensweise an neue Probleme. Jedoch War es einen komplett neue Erfahrung neu Funktionen für echte bezahlende Kunden zu implementieren und somit Einfluss auf die Erfahrung echter Nutzer zu haben.

Auch bekam man einen hautnahen Einblick in Projekt Management und wie es sich auf den Entwicklungsprozess ausübt. Dabei wird deutlich wie sehr sich manchmal das reale entwickeln zu den geplanten Meilensteinen unterscheidet. Das genau Zeitabschätzen für neue geplant Feature stellt sich meist als eigene Herausforderung heraus. Das ist eine Fähigkeit die man nur durch Erfahrung und testen verbessern kann.

Zudem war es einen neue Erfahrung von einem Grafischen Design geleitet, nach Kundenwunsch neu Funktionen zu entwickeln und dabei nur wenig {Mitbestimmungsrecht} zu haben. Während des Studiums kann man Technologien und Design meist selbst bestimmten. In einem Unternehmen gibt es allerdings andere Mitarbeiter und Vorgesetzt die das meiste entscheiden.

\chapter{Ausblick}

Das Praktikum bei Savvi Learning GmbH fungierte als Katalysator für eine beträchtliche Erweiterung der Fähigkeiten im Bereich der Codeerstellung und -optimierung. Durch die direkte Immersion in komplexe und anspruchsvolle Projektlandschaften war eine rasche Adaption und kognitive Agilität erforderlich. Diese initialen Herausforderungen erwiesen sich als didaktische Schlüsselmomente und förderten das technische Verständnis auf ein neues Niveau.

Die erfolgreiche Bewältigung der initialen Hürden führte zu einem signifikanten Kompetenzanstieg im Umgang mit komplexen Codestrukturen und der Entwicklung performanter Lösungen. Diese Erfahrung stärkte nicht nur das Vertrauen in die eigenen Fähigkeiten, sondern schärfte auch die Wahrnehmung der Bedeutung von Flexibilität und der intrinsischen Motivation, in einem dynamischen Arbeitsumfeld kontinuierlich zu lernen.

Mit einem optimistischen Blick auf die Zukunft ist die Motivation ungebrochen, sich neuen und komplexen Aufgabenstellungen zu stellen. Im Rahmen der neuen Funktion als Werkstudent bei Savvi Learning GmbH besteht die große Vorfreude, die erworbenen Kenntnisse und Fähigkeiten kontinuierlich weiterzuentwickeln und einen aktiven Beitrag zum Erfolg des Unternehmens zu leisten.